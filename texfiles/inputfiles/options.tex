
% Ustawienia biblatexa
% Brak przecinka po nazwisku autora w bibliografii
\renewcommand*{\revsdnamepunct}{}

% Kolejność: Nazwisko -> Imie przy wymienianiu autorów
\DeclareNameAlias{sortname}{family-given}

% Zamiana 'and' pomiędzy imionami autorów na 'i'
% Na końcu dokumentu zmiana na 'finalnamedelim' na ',' !!!
\DeclareDelimFormat{finalnamedelim}{\addspace i\addspace}

% Przecinek po ostatnim autorze
\renewcommand*{\nametitledelim}{\addcomma\addspace}

% Zmiana pp. na s.
\DeclareFieldFormat{pages}{s\adddot\addspace #1} % 's. ' w bibliografii
\DeclareFieldFormat{postnote}{s\adddot\addspace #1}
\DeclareFieldFormat{multipostnote}{s\adddot\addspace #1}

% Dodanie w \parentcite przecinek miedzy rokiem a autorem/autorami
\renewcommand*{\nameyeardelim}{\addcomma\addspace}

% Dodane odstępu pomiędzy elementami bibliografii
%\setlength\bibitemsep{2\itemsep}
\setlength\bibitemsep{2\itemsep}

% Usunięcie "In: Journal" w bibliografii
\renewbibmacro{in:}{}

% Dodanie Vol. numer, No numer, w bibliografii
\DeclareFieldFormat[article]{volume}{Vol. \addnbspace #1}
\DeclareFieldFormat[article]{number}{No\addnbspace #1}
\renewbibmacro*{volume+number+eid}{
    \printfield{volume}
    \setunit{\addcomma\space}
    \printfield{number}
    \setunit{\addcomma\space}
    \printfield{eid}
    \setunit{\addcomma\space}
}

% Przecinek po nazwie czasopisma naukowego
\DeclareFieldFormat{journaltitle}{\mkbibemph{#1}\isdot}
\renewbibmacro*{journal+issuetitle}{
    \usebibmacro{journal}
    \setunit*{\addcomma\space}
    \iffieldundef{series}
    {}
    {\newunit
     \printfield{series}
     \setunit{\addspace}}
    \usebibmacro{volume+number+eid}
    \setunit{\addspace}
    \usebibmacro{issue+date}
    \setunit{\addcolon\space}
    \usebibmacro{issue}
    \newunit}    

% Brak nawiasów w roku wydania
\renewbibmacro*{issue+date}{
    \setunit{\addcomma\space}
    \iffieldundef{issue}
    {\usebibmacro{date}}
    {\printfield{issue}
     \setunit*{\addspace}
     \usebibmacro{date}}
\newunit}

% Usunięcie italizacji w bibliografii
\makeatletter
\renewrobustcmd*{\mkbibemph}{}
\protected\long\def\blx@imc@mkbibemph#1{#1}
\makeatother

% Usunięcie cudzysłowów
\DeclareFieldFormat*{title}{#1}

% Wielkość wcięcia w bibliografii
\setlength\bibhang{1cm}

% zmiana 'ibid.' na 'Ibidem'
\DefineBibliographyStrings{english}{
    ibidem = {\textit{Ibidem}},
}

% Rzymska numeracja dla rozdziałów i arabska dla podrozdziałów
\renewcommand{\thesection}{\Roman{section}}
\renewcommand{\thesubsection}{\arabic{section}.\arabic{subsection}}

% Format Rozdziałów i podrozdziałów
\titleformat{\section}[display]{\bfseries\centering}{ROZDZIAŁ \thesection}{5pt}{\centering #1}

% stary format dla powyzszego
%\titleformat{\section}[display]{\bfseries\centering}{ROZDZIAŁ \thesection}{5pt}{#1\quad}

\titleformat{\subsection}[hang]{\bfseries}{\thesubsection.}{5pt}{#1\quad}

% Tytuł 'Bibliografia' wyrównana do lewej strony
\defbibheading{bibliography}[\bibname]{
    \chapter{\raggedleft \textbf{#1}}
    \markboth{#1}{#1}
}

% Potrzebne do zmiany formatu nazw rysunków i wykresów
%\usepackage[figurename=Rys.]{caption}
%\renewcommand{\figurename}{Rys.}
\captionsetup{labelsep=period}

% Centruje "Spis treści" z odległością 2em między tytułem spisu a elementami spisu
\renewcommand{\contentsname}{\hspace*{\fill}\bfseries\Large Spis treści\hspace*{\fill}\vspace{2em}}

% dodaje kropki dla sekcji w spisie tresci + dodaje Rozdział jako przednazwę
\titlecontents{section}
[8em] % szerokość contentslabel (tutaj ROZDZIAŁ)
{\medskip}
{\contentslabel[\MakeUppercase Rozdział \thesection~\thecontentslabel.]{6.5em}}
{\hspace*{-6.5em}}
{\titlerule*[0.8pc]{.}\contentspage}
%{\titlerule*[0.5pc]{.}\contentspage} % najlepiej wygląda


% Alternatywa dla powyższego ale gorsza bo używa paczki "tocloft"
% Dodaje kropki do spisu treści
%\renewcommand{\cftsecleader}{\cftdotfill{\cftdotsep}}

% Zmienia przednazwy rozdziałów w spisie treści
%\let\oldsecpresnum\cftsecpresnum
%\renewcommand\cftsecaftersnum{.}
%\renewcommand{\cftsecpresnum}{ROZDZIAŁ \oldsecpresnum}
%\setlength{\cftsecnumwidth}{8.3em}

