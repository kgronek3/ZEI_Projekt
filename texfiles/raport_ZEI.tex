\documentclass[12pt]{article}
\usepackage[T1]{fontenc}
\usepackage[utf8]{inputenc}
\usepackage{polski}
\usepackage[backend=biber,
            %dashed=false, % usunąć linię przy tym samym autorze
            %style=authoryear-icomp, % jeden styl do cytacji i bibliografii
            citestyle=authoryear-icomp,
            bibstyle=authortitle-icomp,
            maxcitenames=3,
            uniquelist=false, %jeśli ten sam pierwszy autor to nie wymienia drugiego innego
            maxbibnames=100,
            sorting=nyt,
            hyperref=true,
            uniquename=false, % żeby nie drukowało inicjałów dla pracy z tymi samymi nazwiskami
            giveninits=true]{biblatex}
%\usepackage{lmodern} % jeśli kompilowane na windows
%\usepackage[polish]{babel}
%\usepackage[none]{hyphenat} % usunięcie myślników w biblografii
\usepackage[hidelinks]{hyperref} % do klikalnych linkow
\usepackage{geometry} % do formatowania tekstu
\usepackage{setspace} % dla odstępu pomiędzy liniami tekstu
\usepackage[compact,explicit]{titlesec} % dla titleformat
\usepackage{titletoc}
\usepackage{indentfirst} % wcięcie w pierwszym akapicie
\usepackage{amsmath, amsfonts, mathtools, amsthm, amssymb}
\usepackage[figurename=Rys.,singlelinecheck = false, justification = justified, format = hang]{caption} % format tytułów tabel/rysunkow/wykresow

\addbibresource{bibliografia_ZEI.bib}


% Marginesy:
\geometry{
    a4paper,
%    total = {170mm, 257mm},
    left = 25mm,
    right = 25mm,
    top = 25mm,
    bottom = 25mm,
}
% Wcięcie akapitu
\setlength{\parindent}{1cm}

% Odstęp pomiędzy akapitami
\setlength{\parskip}{0.2cm}



% Ustawienia biblatexa
% Brak przecinka po nazwisku autora w bibliografii
\renewcommand*{\revsdnamepunct}{}

% Kolejność: Nazwisko -> Imie przy wymienianiu autorów
\DeclareNameAlias{sortname}{family-given}

% Zamiana 'and' pomiędzy imionami autorów na 'i'
% Na końcu dokumentu zmiana na 'finalnamedelim' na ',' !!!
\DeclareDelimFormat{finalnamedelim}{\addspace i\addspace}

% Przecinek po ostatnim autorze
\renewcommand*{\nametitledelim}{\addcomma\addspace}

% Zmiana pp. na s.
\DeclareFieldFormat{pages}{s\adddot\addspace #1} % 's. ' w bibliografii
\DeclareFieldFormat{postnote}{s\adddot\addspace #1}
\DeclareFieldFormat{multipostnote}{s\adddot\addspace #1}

% Dodanie w \parentcite przecinek miedzy rokiem a autorem/autorami
\renewcommand*{\nameyeardelim}{\addcomma\addspace}

% Dodane odstępu pomiędzy elementami bibliografii
%\setlength\bibitemsep{2\itemsep}
\setlength\bibitemsep{2\itemsep}

% Usunięcie "In: Journal" w bibliografii
\renewbibmacro{in:}{}

% Dodanie Vol. numer, No numer, w bibliografii
\DeclareFieldFormat[article]{volume}{Vol. \addnbspace #1}
\DeclareFieldFormat[article]{number}{No\addnbspace #1}
\renewbibmacro*{volume+number+eid}{
    \printfield{volume}
    \setunit{\addcomma\space}
    \printfield{number}
    \setunit{\addcomma\space}
    \printfield{eid}
    \setunit{\addcomma\space}
}

% Przecinek po nazwie czasopisma naukowego
\DeclareFieldFormat{journaltitle}{\mkbibemph{#1}\isdot}
\renewbibmacro*{journal+issuetitle}{
    \usebibmacro{journal}
    \setunit*{\addcomma\space}
    \iffieldundef{series}
    {}
    {\newunit
     \printfield{series}
     \setunit{\addspace}}
    \usebibmacro{volume+number+eid}
    \setunit{\addspace}
    \usebibmacro{issue+date}
    \setunit{\addcolon\space}
    \usebibmacro{issue}
    \newunit}    

% Brak nawiasów w roku wydania
\renewbibmacro*{issue+date}{
    \setunit{\addcomma\space}
    \iffieldundef{issue}
    {\usebibmacro{date}}
    {\printfield{issue}
     \setunit*{\addspace}
     \usebibmacro{date}}
\newunit}

% Usunięcie italizacji w bibliografii
\makeatletter
\renewrobustcmd*{\mkbibemph}{}
\protected\long\def\blx@imc@mkbibemph#1{#1}
\makeatother

% Usunięcie cudzysłowów
\DeclareFieldFormat*{title}{#1}

% Wielkość wcięcia w bibliografii
\setlength\bibhang{1cm}

% zmiana 'ibid.' na 'Ibidem'
\DefineBibliographyStrings{english}{
    ibidem = {\textit{Ibidem}},
}

% Rzymska numeracja dla rozdziałów i arabska dla podrozdziałów
\renewcommand{\thesection}{\Roman{section}}
\renewcommand{\thesubsection}{\arabic{section}.\arabic{subsection}}

% Format Rozdziałów i podrozdziałów
\titleformat{\section}[display]{\bfseries\centering}{ROZDZIAŁ \thesection}{5pt}{\centering #1}

% stary format dla powyzszego
%\titleformat{\section}[display]{\bfseries\centering}{ROZDZIAŁ \thesection}{5pt}{#1\quad}

\titleformat{\subsection}[hang]{\bfseries}{\thesubsection.}{5pt}{#1\quad}

% Tytuł 'Bibliografia' wyrównana do lewej strony
\defbibheading{bibliography}[\bibname]{
    \chapter{\raggedleft \textbf{#1}}
    \markboth{#1}{#1}
}

% Potrzebne do zmiany formatu nazw rysunków i wykresów
%\usepackage[figurename=Rys.]{caption}
%\renewcommand{\figurename}{Rys.}
\captionsetup{labelsep=period}

% Centruje "Spis treści" z odległością 2em między tytułem spisu a elementami spisu
\renewcommand{\contentsname}{\hspace*{\fill}\bfseries\Large Spis treści\hspace*{\fill}\vspace{2em}}

% dodaje kropki dla sekcji w spisie tresci + dodaje Rozdział jako przednazwę
\titlecontents{section}
[8em] % szerokość contentslabel (tutaj ROZDZIAŁ)
{\medskip}
{\contentslabel[\MakeUppercase Rozdział \thesection~\thecontentslabel.]{6.5em}}
{\hspace*{-6.5em}}
{\titlerule*[0.8pc]{.}\contentspage}
%{\titlerule*[0.5pc]{.}\contentspage} % najlepiej wygląda


% Alternatywa dla powyższego ale gorsza bo używa paczki "tocloft"
% Dodaje kropki do spisu treści
%\renewcommand{\cftsecleader}{\cftdotfill{\cftdotsep}}

% Zmienia przednazwy rozdziałów w spisie treści
%\let\oldsecpresnum\cftsecpresnum
%\renewcommand\cftsecaftersnum{.}
%\renewcommand{\cftsecpresnum}{ROZDZIAŁ \oldsecpresnum}
%\setlength{\cftsecnumwidth}{8.3em}



\begin{document}

\begin{titlepage}
    \begin{center}
        %\vspace{1cm}
            
        {\Large
        Uniwersytet Warszawski\\
        Wydział Nauk Ekonomicznych}
        \vspace{3cm}
        
        Krystian Gronek\\
        Nr albumu: 403706
            
        \vspace{1cm}
           
        {\Large
        \textbf{Tytuł pracy}}
           
        \vspace{1.5cm}
        Praca licencjacka \\
        na kierunku: Informatyka i Ekonometria
    \end{center}
        \vspace{3cm}
    \begin{flushright}
        Praca wykonana pod kierunkiem\\
        PROWADZĄCY\\
        z Katedry KATEDRA\\
        WNE UW
    \end{flushright}
        \vfill
    \begin{center}
        Warszawa, MIESIĄC 2023
    \end{center}
\end{titlepage}

\vspace*{6\baselineskip}
\begin{center}
\textbf{Streszczenie}
\end{center}
abstrakt powinien składać się z jednego-dwóch akapitów. proszę wypisać główne cele pracy, napisać jaką metodą lub modelem weryfikowano hipotezę lub modelowano zjawisko oraz jakie są wnioski z pracy. proszę przeczytać: https://pl.wikipedia.org/wiki/abstrakt lub https://www.springer.com/ (link)
\vspace{4\baselineskip}
\begin{center}
\textbf{Słowa kluczowe}
\end{center}
\begin{center}
słowo klucz 1, słowo klucz 2 
\end{center}
\vspace{3\baselineskip}
\begin{center}
\textbf{Dziedzina pracy}
\end{center}
\begin{center}
Ekonomia (14300)
\end{center}
\vspace{3\baselineskip}
\begin{center}
\textbf{Klasyfikacja tematyczna}
\end{center}
\vspace{4\baselineskip}
\begin{center}
\textbf{Tytuł pracy w języku angielskim}
\end{center}
\begin{center}
TITLE IN ENGLISH
\end{center}

\newpage


\tableofcontents

\newpage

% Odstęp pomiędzy wierszami wewnątrz akapitu
\spacing{1.5}

\addcontentsline{toc}{section}{WSTĘP}
\subsection*{WSTĘP}
We wstępie należy opisać problem/zagadnienie. Przedstawić hipotezy badawcze. Uzasadnić hipotezy badawcze. Napisać dlaczego temat jest ważny, dlaczego warto się nim zajmować, komu wnioski mogą być przydatne. We wstępie można przedstawić metody i wnioski z literatury.

\newpage 

\section{Przegląd literatury}
W części poświęconej przeglądowi literatury zawrzeć artykuły na jakich opierałem swoją pracę oraz metody, dane i zagadnienia jakie one badały.

\section{Dane}
W części poświęconej danym proszę opisać dane, napisać skąd pochodzą oraz jakim, jeśli jakimś, przekształceniom zostały poddane. Jeśli usunęliście Państwo jakieś obserwacje, to także proszę o tym napisać. Wykres danych, histogram lub jakaś forma wizualizacji danych może być pomocna dla czytelnika Państwa pracy do zrozumienia problemu lub hipotezy.

\section{Model}
Proszę opisać model lub metodę, której Państwo używacie w zrozumiały sposób, także dla osoby, która nie zna metody która Państwo się posługujecie. Idealnie byłoby, gdybyście Państwo tak opisali swoją metodę, aby każdy kto chciałby powtórzyć Państwa badanie nie miał z tym problemów. Proszę uzasadnić wybór metody. Proszę napisać jakie są potencjalne lub rzeczywiste wady i zalety metody.

\section{Wyniki}
W rozdziale wyniki proszę zweryfikować hipotezy badawcze. Podać prawdopodobne uzasadnienie w przypadku odrzucenia badanych hipotez.

\addcontentsline{toc}{section}{ZAKOŃCZENIE}
\subsection*{ZAKOŃCZENIE}
Rozdział „wnioski” powinien zawierać powtórzenie najważniejszych wniosków oraz wskazanie kierunków dalszych prac nad tym tematem.

\newpage


\newpage

% to tutaj potrzebne aby na końcu było bez 'i' pomiędzy autorami
%\renewcommand*{\finalnamedelim}{\addcomma\addspace}

% Zmiana nazwy bibliografii z 'References' na 'Bibliografia'
\addcontentsline{toc}{section}{BIBLIOGRAFIA}
\printbibliography[title ={BIBLIOGRAFIA}]

\newpage

\addcontentsline{toc}{section}{ZESTAWIENIE SPISÓW}
\subsection*{ZESTAWIENIE SPISÓW}

\addcontentsline{toc}{section}{ZESTAWIENIE TABEL}
\listoftables

\addcontentsline{toc}{section}{ZAŁĄCZNIKI}
\listoffigures

\newpage


\newpage

\end{document}

